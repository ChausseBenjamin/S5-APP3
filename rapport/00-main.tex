\documentclass[a4paper]{article}
\usepackage{titling}
\usepackage{tabularx}
\usepackage{titlepage}
\usepackage{amsmath}
\usepackage{amsthm}
\usepackage{document}
\usepackage{booktabs}
\usepackage{float}
\usepackage{graphicx}
\usepackage{utils}
\usepackage{multicol}
\usepackage{subcaption}
\usepackage[dvipsnames]{xcolor}
\usepackage{enumitem}
\usepackage{siunitx}
\usepackage{hyperref}
\usepackage{times}
\usepackage[toc,page]{appendix}

% Guide etudiant requires custom header/footer for some reason
\usepackage{fancyhdr}

\title{Rapport d'APP}

\class{Traitement numérique des signaux}
\classnb{GIF570}
\renewcommand{\frenchtablename}{Tableau}

\teacher{Philippe Gourmay}

\author{
  \addtolength{\tabcolsep}{-0.4em}
  \begin{tabular}{rcl}
      Benjamin Chausse & -- & CHAB1704 \\
      Shawn Couture    & -- & COUS1912 \\
  \end{tabular}
}

\makeatletter
\let\title\@title
\let\authors\@author
\let\thedate\@date
\makeatother

\begin{document}
\maketitle
\contents{
\tableofcontents
\listoffigures
\listoftables
}
\pagestyle{fancy}
\fancyhf{} % clear default header/footer
\fancyhead[L]{\footnotesize \title}
\fancyhead[R]{\footnotesize Benjamin Chaussé, Shawn Couture}
\fancyfoot[C]{\thepage}
\renewcommand{\headrulewidth}{0.4pt} % optional
\renewcommand{\footrulewidth}{0pt}


\section{Introduction}

Ce rapport présente l’étude des signaux sonores pour un dispositif d’interaction
homme-machine chez AnthRoBotik. La première partie traite l’analyse et la
synthèse d'une note de guitare, en extrayant les sinusoïdes et l'enveloppe
temporelle pour générer des sons modulables. La seconde partie se concentre sur
le débruitage et la restauration du signal audio d'un basson corrompu via des
filtres numériques.

\section{Analyse et synthèse}
\subsection{Analyse du signal}

\subsubsection{Méthode d'extraction}
La première chose à faire avec le signal d'origine est d'en extraire les paramètres audio.
En effet, pour synthétisé le signal, les paramètres du signal d'origine sont nécessaire.
Ces paramètres sont le spectre fréquentielle (amplitude de chaque fréquence), phase de
chaque fréquence et l'enveloppe temporelle du signal.

Les amplitudes et phases sont obtenue à l'aide d'une analyze Fourier (FFT) sur le signal
d'origine, en quelques fonctions python. L'enveloppe temporelle du signal est obtenue en
convoluant un filtre passe-bas à moyenne mobile, de type RIF, avec l'absolue du signal 
d'origine.

Ce filtre est nécessaire car l'enveloppe temporelle représente une fonction de l'amplitude
à travers le temps et non une oscillation. Mettre le signal en valeur absolue permet d'avoir
une moyenne qui n'est pas 0. De plus ceci créé quasiment une enveloppe tel quel. L'oscillation 
est proche d'une enveloppe temporelle, mais il reste des composantes à haute fréqunces.
Le filtre passe-bas permet donc d'enlevé ces hautes fréquences et faire une moyenne sur 
des amplitudes et donc obtenir un graphique lisse sans fréquences risiduelle.
Le filtre à moyenne mobile est choisi car il est linéaire et causale, donc n'introduit pas de
déformations. La fréquence de coupure est choisi à $\pi/1000$ dû à la variation très lente du
signal d'origine. L'ordre N du filtre est expliqué dans sa propre section.

\subsubsection{Filtre RIF}
La réponse impulsionnelle d'un filtre RIF à moyenne mobile à une fréquence normalizé est comme suit:
$$
H=\frac{1}{N}\sum^{N-1}_{n=0}e^{-j\pi\bar\omega}
$$
Il est possible d'utilisé la série géométrique pour enlever la somme, mais Python permet déjà le calcul
tel quelle. Hors, la fréquence à utilisé est $\pi/1000$. Hors, suffi de tester tout les ordres possibles
pour obtenir l'ordre le plus près d'un gain de -3dB pour la fréquence. Ce test à été fait en utilisant un
ordre entre un minimum et un maximum. Si le gain est trop bas, un nouveau minimum est choisi, vise versa,
jusqu'a ce que la différence entre l'ordre maximum et l'ordre minimum soit plus bas ou égal à 1. L'ordre
N choisi de cette façon est $885$.
\subsection{Synthèse des sons}


\section{Débruitage et restauration du signal}

Cette section présente la restauration du signal corrompu du basson. L’objectif
est d’éliminer une sinusoïde parasite à \SI{1}{\kilo\hertz} tout en préservant
les caractéristiques essentielles du signal original. La méthode choisie
consiste à appliquer un filtrage numérique très localisé afin de supprimer la
composante indésirable sans altérer (ou presque) le reste des fréquences du
signal.

Pour atteindre cet objectif, un filtre RIF passe-bas est d’abord conçu par la
méthode de la fenêtre, puis transformé en filtre coupe-bande centré sur la
fréquence parasite. Le filtre initial est défini avec un ordre $N=6000$ et une
bande de coupure de ±\SI{40}{\hertz}, ce qui permet de couvrir l’incertitude sur
la fréquence exacte de la sinusoïde. Les équations suivantes résument la
conception du filtre et sa conversion en filtre coupe-bande, qui sera ensuite
convolué avec le signal d’origine pour restaurer le basson.

\begin{equation}
k = \frac{2 \cdot f_c \cdot N}{f_s} + 1
  = \frac{2 \cdot \SI{40}{\hertz} \cdot 6000}{\SI{44100}{\hertz}} + 1
  = \frac{480000}{44100} + 1
  \approx 10.88 + 1 = 11.88
\end{equation}

L'équation du passe-bas de départ devient alors:

\begin{equation}
	h_{lp}[n] = \frac{1}{N}\cdot \frac{ \sin{(n\pi k/N)} }{ \sin{(n\pi/N)} }
	          = 6000^{-1} \frac{ \sin{(n\pi \cdot 11.88/6000)} }{ \sin{(n\pi/6000)} }
\end{equation}

Toutefois, le but ultime est de couper toutes les fréquence à $f =
\SI{1}{\kilo\hertz}$. Il faut donc trouver le décalage requis en terme de
trames/bins requis ($\omega_0$) pour que le filtre affecte la fréquence voulue
$f$:

\begin{equation}
	\omega_0 = 2\pi\frac{f}{f_s}
					 = 2\pi\frac{\SI{1000}{\hertz}}{\SI{44100}{\hertz}}
					 = \frac{2000\pi}{44100}
					 \approx 0.1425
\end{equation}

Le passe-bas peut ensuite être transformé en coupe-bande (dont la réponse
impulsionelle peut être observée à la figure \ref{fig:bassoon-notch}) comme suit en utilisant
des tables de conversion précédemment fournies:

\begin{align}
	h_{bs}[n] &= \delta[n]-2h_{lp}[n]\cos{(\omega_0n)} \\
	          &= \delta[n]-2 * 6000^{-1}
						   \frac{
							   \sin{(n\pi \cdot 11.88/6000)} }{
								 \sin{(n\pi/6000)}
							 } \cos{(\omega_0n)} \\
	          &= \delta[n]-2 * 6000^{-1}
						   \frac{
							   \sin{(n\pi \cdot 11.88/6000)} }{
								 \sin{(n\pi/6000)}
							 } \cos{(\frac{2000n\pi}{44100})}
\end{align}

Ce coupe-bande est enfin convolué avec le signal d'origine pour générer une
version assainie du signal. Le fonctionnement du filtre peut être confirmé en
comparant la figure \ref{fig:bassoon-cleaned} à la figure \ref{fig:bassoon-raw}.
Un mur de son constant, observable à la sous-figure \ref{fig:bassoon-raw-sig},
disparaît dans \ref{fig:bassoon-cleaned-sig} (à l'exception de la période
transitoire de début et de fin). De plus, les pics spectraux à
$\pm\SI{1}{\kilo\hertz}$ (le $\pm$ dénote que la figure contient les fréquences
négatives pour conserver la symétrie), causés par la sinusoïde parasite en
\ref{fig:bassoon-raw-freq}, disparaissent complètement, comme le montre la
sous-figure \ref{fig:bassoon-cleaned-freq}.

La réponse impulsionnelle h[n] du filtre coupe-bande résultant peut être
observée à la figure \ref{fig:bassoon-notch-impulse}, qui illustre la forme
caractéristique de ce type de filtre avec un pic central entouré d'oscillations
amorties.

L'efficacité du filtre peut être vérifiée en observant sa réponse à une
sinusoïde pure de \SI{1000}{\hertz} (figure \ref{fig:bassoon-test-with-1k}).
Cette expérience montre que le signal sinusoïdal est effectivement supprimé
par le filtre coupe-bande, confirmant son bon fonctionnement à la fréquence
cible.


Une fois le code rédigé dans \textit{Python} et ce nouveau signal créé, il
devient évident aux membres de l'équipe que le filtre conçu n'arrive pas à
éliminer la fréquence parasite dans sa totalité. Bien que le guide étudiant
ait suggéré la réexécution du filtre plusieurs fois pour en amplifier ses
effets, quelques expériences sont faites pour tenter de minimiser la
corruption du signal du basson. Cette expérimentation mène à la réalisation
que bien qu'une fréquence de coupure de $\pm\SI{40}{\hertz}$ soit exigée,
rétrécir la plage de coupure à $\pm\SI{1}{\hertz}$ et n'appliquer le filtre
qu'une seule fois a une incidence bien moindre sur l'intégrité du signal.
Bien que cette expérience soit de nature subjective, elle peut être expliquée
par le fait qu'un filtre plus large laisse passer une plus grande part de
l'énergie de la fréquence indésirable. Cependant, pour respecter les exigences
du laboratoire, les résultats présentés utilisent la spécification originale
de $\pm\SI{40}{\hertz}$ avec un seul passage du filtre.


\section{Conclusion}

La synthèse et le débruitage réalisés dans ce rapport montrent que les signaux
audio peuvent être reconstruits et assainis avec précision en utilisant des
méthodes numériques adaptées. L’extraction des paramètres et harmoniques a
permis de reproduire fidèlement les sons de guitare. Le filtrage coupe-bande a
efficacement réduit la sinusoïde parasite du basson tout en préservant
l’intégrité du signal original. Ces techniques illustrent l’efficacité des
traitements numériques pour l’analyse et la manipulation sonore dans des
contextes homme-machine. L’approche proposée fournit une base solide pour des
applications audio modulables et contrôlées.

\newpage
\appendix
\section{Traitement de la guitare}

\todo{
Affichage des spectres de Fourier des signaux du LA# et du basson (originaux)
en dB avec l’axe des fréquences en Hz
}



\begin{table}[H]
	\centering
	\caption{Harmoniques de la note de guitare}
	\label{tab:guitar-harmonics}
	\footnotesize{
	\begin{tabular}{llrr}
		\toprule
		Index & Amp. (dB) & Fréq. (\si{\hertz}) & Phase (rad) \\
		\midrule
		 1  & 154.15 &   467.18 & -2.9766 \\
		 2  & 156.03 &   931.34 &  1.0890 \\
		 3  & 151.76 &  1397.97 &  0.0823 \\
		 4  & 139.74 &  1864.88 & -1.4838 \\
		 5  & 140.01 &  2329.31 & -1.7659 \\
		 6  & 141.38 &  2797.87 &  0.3606 \\
		 7  & 134.74 &  3264.78 & -1.2277 \\
		 8  & 131.15 &  3726.45 & -2.6815 \\
		 9  & 131.23 &  4195.56 & -1.6767 \\
		 10 & 124.00 &  4662.20 & -2.8018 \\
		 11 & 125.75 &  5126.62 & -2.7569 \\
		 12 & 122.94 &  5593.26 &  2.3990 \\
		 13 & 120.42 &  6074.22 &  0.3645 \\
		 14 & 119.37 &  6524.32 &  1.2354 \\
		 15 & 118.91 &  7025.96 & -2.5247 \\
		 16 & 116.18 &  7484.87 &  0.7796 \\
		 17 & 119.23 &  7956.74 &  0.6406 \\
		 18 & 114.13 &  8446.25 & -0.1434 \\
		 19 & 115.24 &  8920.60 & -3.0627 \\
		 20 & 115.00 &  9376.76 & -2.9300 \\
		 21 & 114.72 &  9851.66 & -0.6240 \\
		 22 & 109.52 & 10250.22 &  3.0618 \\
		 23 & 111.49 & 10829.58 &  0.3800 \\
		 24 & 105.56 & 11185.14 & -2.6147 \\
		 25 & 110.48 & 11790.69 & -2.9863 \\
		 26 & 110.63 & 12273.03 & -2.3882 \\
		 27 & 111.14 & 12721.75 & -0.5966 \\
		 28 & 110.56 & 13203.82 &  0.6351 \\
		 29 & 107.24 & 13726.40 &  2.9570 \\
		 30 & 102.87 & 14171.53 &  1.5487 \\
		 31 & 106.57 & 14657.46 & -0.7994 \\
		\bottomrule
\end{tabular} }
\end{table}

\section{Traitement du basson}


\begin{figure}[H]
  \centering
	\begin{subfigure}{\linewidth}
		\centering
		\includegraphics[width=.85\textwidth]{figures/bassoon-raw-sig.pdf}
		\caption{Représentation temporelle}
		\label{fig:bassoon-raw-sig}
	\end{subfigure}
	\begin{subfigure}{\linewidth}
		\centering
		\includegraphics[width=.85\textwidth]{figures/bassoon-raw-freq.pdf}
		\caption{Représentation fréquencielle}
		\label{fig:bassoon-raw-freq}
	\end{subfigure}

	\caption{Analyse du basson d'origine non-traité}
	\label{fig:bassoon-raw}
\end{figure}

\vfill

\begin{figure}[H]
  \centering
	\begin{subfigure}{\linewidth}
		\centering
		\includegraphics[width=.85\textwidth]{figures/bassoon-cleaned-sig.pdf}
		\caption{Représentation temporelle}
		\label{fig:bassoon-cleaned-sig}
	\end{subfigure}
	\begin{subfigure}{\linewidth}
		\centering
		\includegraphics[width=.85\textwidth]{figures/bassoon-cleaned-freq.pdf}
		\caption{Représentation fréquencielle}
		\label{fig:bassoon-cleaned-freq}
	\end{subfigure}

	\caption{Analyse du basson assaini}
	\label{fig:bassoon-cleaned}
\end{figure}

\vfill

\begin{figure}
\centering
	\begin{subfigure}{\linewidth}
		\centering
		\includegraphics[width=.85\textwidth]{figures/bassoon-notch-freq.pdf}
		\caption{Amplitude (dB)}
		\label{fig:bassoon-notch-freq}
	\end{subfigure}
	\begin{subfigure}{\linewidth}
		\centering
		\includegraphics[width=.85\textwidth]{figures/bassoon-notch-phase.pdf}
		\caption{Phase (rad)}
		\label{fig:bassoon-notch-phase}
	\end{subfigure}

	\caption{Réponse impulsionelle du filtre coupe-bande}
	\label{fig:bassoon-notch}
\end{figure}



\end{document}
