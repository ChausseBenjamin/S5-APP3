\subsection{Analyse du signal}

\subsubsection{Méthode d'extraction}
La première chose à faire avec le signal d'origine est d'en extraire les paramètres audio.
En effet, pour synthétisé le signal, les paramètres du signal d'origine sont nécessaire.
Ces paramètres sont le spectre fréquentielle (amplitude de chaque fréquence), phase de
chaque fréquence et l'enveloppe temporelle du signal.

Les amplitudes et phases sont obtenue à l'aide d'une analyze Fourier (FFT) sur le signal
d'origine, en quelques fonctions python. L'enveloppe temporelle du signal est obtenue en
convoluant un filtre passe-bas à moyenne mobile, de type RIF, avec l'absolue du signal 
d'origine.

Ce filtre est nécessaire car l'enveloppe temporelle représente une fonction de l'amplitude
à travers le temps et non une oscillation. Mettre le signal en valeur absolue permet d'avoir
une moyenne qui n'est pas 0. De plus ceci créé quasiment une enveloppe tel quel. L'oscillation 
est proche d'une enveloppe temporelle, mais il reste des composantes à haute fréqunces.
Le filtre passe-bas permet donc d'enlevé ces hautes fréquences et faire une moyenne sur 
des amplitudes et donc obtenir un graphique lisse sans fréquences risiduelle.
Le filtre à moyenne mobile est choisi car il est linéaire et causale, donc n'introduit pas de
déformations. La fréquence de coupure est choisi à \pi/1000 dû à la variation très lente du
signal d'origine. L'ordre N du filtre est expliqué dans sa propre section.

