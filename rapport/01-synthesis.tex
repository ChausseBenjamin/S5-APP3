\subsection{Synthèse des sons}
Pour faire la synthèse d'un son, il faut ses paramètres, qui sont obtenue à la section précédente.
Cependant, il faut aussi les harmoniques.

\subsubsection{Obtenir les harmoniques}
Les harmoniques sont visible en affichant le graphique des amplitudes des fréquences en décibel.
Ils sont plus élevé que les autres fréquences du signal. Cependant, aller chercher directement
les plus grands amplitudes ne fonctionne pas, il faut le faire par groupe locaux afin d'être sur
de n'avoir que les harmoniques. Hors, python à une fonction préfaite qui permet de les obtenir et
de mettre une distance minimum entre les maximums pour être sur d'obtenir que les harmoniques.
En ajustant cette valeur de distance (1285) les 31 premières harmoniques ont pu être obtenue sur
le signal de guitar. La 32e n'était pas à la bonne place, et donc à été enlevé.
La première harmonique correspond à la fondamental du signal, qui à été utilisé lors de la synthèse.

Une fonction python pour obtenir les harmoniques à partir d'un signal à été créé, et est représenté
dans le schéma bloc. La figure \ref{fig:31-premières-harmoniques} montre les 31 premières harmoniques
de la guitar.

\subsubsection{Synthèse du son}
Plusieurs méthodes sont possible. La toute première fût avec une somme de nombres imaginaires qui
correspondait à la réponse impulsionnelle, ça fonctionnais mais c'était très lent. Le signal de
synthèse peut être aussi créé en fesant la transformée de Fourier inverse du spectre fréquentielle
des harmoniques pour obtenir un signal qui peut être multiplier par l'enveloppe temporelle. La
méthode choisi est encore plus simple à comprendre programmatiquement. Un tableau de sinus est
généré en leurs donnant le tableau de phases, fréquences et amplitudes. Ce dernier est directement
multiplier par l'enveloppe temporelle pour avoir la bonne amplitude à travers le temps.
Par la suite, le signal synthétisé est multiplier par un ratio afin qu'il ait la même amplitude
maximal que le signal d'origine. Ce ratio est un rapport entre l'amplitude maximal de la synthèse
et le signal d'origine. La figure \ref{fig:guitar-original-versus-synthétisé} montre la différence
fréquentielle entre le signau d'origine et le synthétisé.

\subsubsection{Génération des autres notes}
La même exact méthode que la sous-section précédente est utilisé, sauf que cette fois-ci, la
fréquence de tout les sinus est multiplier par un facteur $k$ autre que $1$. Ce facteur est
obtenue en fesant le ratio de la fréquence fondamental du signal d'origine et la fréquence voulue,
ce qui dicte de combien elle doit être muiltiplier afin d'obtenir la note voulue. Voici l'équation
d'une sinus
$$
\text{amplitude}\cdot\sin(2\pi\cdot\text{fréquence}\cdot k+\text{phase})
$$