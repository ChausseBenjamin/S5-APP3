
Cette section présente la restauration du signal corrompu du basson. L’objectif
est d’éliminer une sinusoïde parasite à \SI{1}{\kilo\hertz} tout en préservant
les caractéristiques essentielles du signal original. La méthode choisie
consiste à appliquer un filtrage numérique très localisé afin de supprimer la
composante indésirable sans altérer (ou presque) le reste des fréquences du
signal.

Pour atteindre cet objectif, un filtre RIF passe-bas est d’abord conçu par la
méthode de la fenêtre, puis transformé en filtre coupe-bande centré sur la
fréquence parasite. Le filtre initial est défini avec un ordre $N=6000$ et une
bande de coupure de ±\SI{40}{\hertz}, ce qui permet de couvrir l’incertitude sur
la fréquence exacte de la sinusoïde. Les équations suivantes résument la
conception du filtre et sa conversion en filtre coupe-bande, qui sera ensuite
convolué avec le signal d’origine pour restaurer le basson.

\begin{equation}
f_c = \frac{k-1}{2} \Rightarrow
k   = 2f_c + 1
    = 2\cdot\SI{40}{\hertz} + 1
    = 81
\end{equation}

L'équation du passe-bas de départ devient alors:

\begin{equation}
	h_{lp}[n] = \frac{1}{N}\cdot \frac{ \sin{(n\pi k/N)} }{ \sin{(n\pi/N)} }
	          = 6000^{-1} \frac{ \sin{(n\pi81/6000)} }{ \sin{(n\pi/6000)} }
\end{equation}

Toutefois, le but ultime est de couper toutes les fréquence à $f_c =
\SI{1}{\kilo\hertz}$. Il faut donc trouver le décalage requis en terme de
trames/bins requis ($\omega_0$) pour que le filtre affecte la fréquence voulue
$f_c$:

\begin{equation}
	\omega_0 = 2\pi\frac{f_c}{f_s}
					 = 2\pi\frac{\SI{1}{\kilo\hertz}}{\SI{44.1}{\kilo\hertz}}
					 = \frac{2\pi}{44.1}
					 \approx 0.1425
\end{equation}

Le passe-bas peut ensuite être transformé en coupe-bande (dont la réponse
impulsionelle peut être observée à la figure \ref{fig:bassoon-notch}) comme suit en utilisant
des tables de conversion précédemment fournies:

\begin{align}
	h_{bs}[n] &= \delta[n]-2h_{lp}[n]\cos{(\omega_0n)} \\
	          &= \delta[n]-2 * 6000^{-1}
						   \frac{
							   \sin{(n\pi\SI{81}{\hertz}/6000)} }{
								 \sin{(n\pi/6000)}
							 } \cos{(\omega_0n)} \\
	          &= \delta[n]-2 * 6000^{-1}
						   \frac{
							   \sin{(n\pi\SI{81}{\hertz}/6000)} }{
								 \sin{(n\pi/6000)}
							 } \cos{(\frac{2n\pi}{44.1})}
\end{align}

Ce coupe-bande est enfin convolué avec le signal d'origine pour générer une
version assainie du signal. Le fonctionnement du filtre peut être confirmé en
comparant la figure \ref{fig:bassoon-cleaned} à la figure \ref{fig:bassoon-raw}.
Un mur de son constant, observable à la sous-figure \ref{fig:bassoon-raw-sig},
disparaît dans \ref{fig:bassoon-cleaned-sig} (à l'exception de la période
transitoire de début et de fin). De plus, les pics spectraux à
$\pm\SI{1}{\kilo\hertz}$ (le $\pm$ dénote que la figure contient les fréquences
négatives pour conserver la symétrie), causés par la sinusoïde parasite en
\ref{fig:bassoon-raw-freq}, disparaissent complètement, comme le montre la
sous-figure \ref{fig:bassoon-cleaned-freq}.

\todo{Ajouter l'équation aux différences du filtre coupe-bande et expliquer le
calcul des valeurs des coefficients}

\todo{Ajouter un graphique de la réponse impulsionnelle h(n) du filtre
coupe-bande}

\todo{Ajouter un graphique montrant la réponse du filtre à une sinusoïde de 1000
Hz pour démontrer son efficacité}


Une fois le code rédigé dans \textit{python} et ce nouveau signal créé, il
devient évident aux membres de l'équipe que le filtre conçu n'arrive pas à
éliminer la fréquence parasite dans sa totalité. Bien que le guide étudiant
ait suggère la réexécution du filtre plusieurs fois pour en amplifier ses
effets, quelques expériences sont faites pour tenter de minimiser la
coruption du signal du basson. Cette expérimentation mène à la réalisation
que bien qu'une fréquence de coupure de $\pm\SI{40}{\hertz}$ soit exigé,
rétrécir la plage de coupure à $\pm\SI{1}{\hertz}$ et n'appliquer le filtre
qu'une seule fois a une incidence bien moindre sur l'intégrité du signal.
Bien que cette expérience soit de nature subjective, elle peut être expliquée
par le fait qu'un filtre plus large laisse passer une plus grande part de
l'énergie de la fréquence indésirable.
